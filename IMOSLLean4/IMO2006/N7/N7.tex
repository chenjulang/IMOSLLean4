Prove that for any $b \in \N$ and $n \in \N^+$, there exists $m \in \N$ such that $n \mid b^m + m$.



\subsection*{Solution}

Official solution: \url{https://www.imo-official.org/problems/IMO2006SL.pdf}

We follow the official solution, which proves something stronger.
However, considering $d = \gcd(k, n)$ is \emph{not} necessary at all.
We will not use $d$ in this solution.

We claim that for any $n \in \N^+$ and $N, r \in \N$, there exists $m \geq N$ such that $b^m + m \equiv r \pmod{n}$.
We proceed by induction on $n$, with the base case $n = 1$ obvious.
Now fix some $n$ and suppose the result holds for all smaller $n$.

By pigeonhole principle, there exists $0 \leq M < M + k \leq n$ such that $b^{M + k} \equiv b^M \pmod{n}$.
In fact, $k < n$ is possible; $k = 1$ works if $n$ is a power of $b$.
Otherwise there are at most $n - 1$ residues to deal with and thus some $k < n$ works by pigeonhole principle.

Now fix $N, r \in \N$.
By induction hypothesis, there exists $m' \geq \max\{M, N\}$ such that $b^{m'} + m' \equiv r \pmod{k}$.
Then there exists some $t \in \N$ such that $b^{m'} + m' \equiv r + kt \pmod{n}$.\footnote{
    This does not require $k \mid n$, which is the reason using $\gcd(k, n)$ is unnecessary.
    Admittedly, using $\gcd(k, n)$ gives a less confusing proof.}
Note that $b^{m' + k} \equiv b^{m'} \pmod{n}$ since $m' \geq M$.
By induction, we get $b^{m' + ks} \equiv b^{m'} \pmod{n}$ for any $s \in \N$.
Now choose any $s$ such that $k \mid t + s$.
Taking $m = m' + ks$ gives 
\[ b^m + m \equiv b^{m'} + (m' + sk) \equiv r + k(t + s) \equiv r \pmod{n}. \]
Induction step is complete.
The claim has been proved.



\subsection*{Extra notes}

The original version only considers $b = 2$.
However, the proof works for all values of $b$.
