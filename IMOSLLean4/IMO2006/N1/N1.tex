Find all pairs $(x, y) \in \N \times \Z$ such that $2^{2x + 1} + 2^x + 1 = y^2$.



\subsection*{Answer}

$(0, \pm 2)$ and $(4, \pm 23)$.



\subsection*{Solution}

Official solution: \url{https://www.imo-official.org/problems/IMO2006SL.pdf}

We present the official solution.

Assume that $x > 0$ and $y \geq 0$; the case $x = 0$ is easy to bash.
Considering the equation mod $8$ gives $x \geq 3$.
Write $x = x' + 2$ for some $x' > 0$.

Rearranging gives
\[ (2^{x' + 3} + 1) 2^{x' + 2} = (2^{x + 1} + 1) 2^x = (y - 1)(y + 1). \]
Then $y$ is odd, and one of $y - 1$ or $y + 1$ is exactly $2 \pmod{4}$.
The other must be divisible by $2^{x' + 1}$, so $y = 2^{x' + 1} m + \varepsilon$ for some $\varepsilon = \pm 1$.
In both cases, $y \geq 0$ yields $m \geq 0$.

If $\varepsilon = 1$, then rearranging the equation again gives
\[ 2^{x' + 3} + 1 = (2^{x'} m + 1) m \implies 8 \cdot 2^{x'} + 1 = m^2 2^{x'} + m. \]
Clearly, $m \geq 3$ yields a contradiction, so $m \leq 2$.
The case $m = 0$ and $m = 2$ yields a contradiction by parity, while $m = 1$ yields $7 \cdot 2^{x'} = 0$, another contradiction.

If $\varepsilon = -1$, then rearranging the equation again gives
\[ 2^{x' + 3} + 1 = (2^{x'} m - 1) m \implies 8 \cdot 2^{x'} + 1 = m^2 2^{x'} - m \implies 2^{x'} (m^2 - 8) = m + 1. \]
If $m < 3$, then the LHS is negative while the RHS is positive.
If $m > 3$, then $m^2 - 8 > m + 1 > 0$; a contradiction.
Thus, we get $m = 3$, and rearranging now gives $2^{x'} = 4 \iff x' = 2$.
Recovering $x$ and $y$ gives $x = x' + 2 = 4$ and $y = 2^{x' + 1} m - 1 = 2^3 \cdot 3 - 1 = 23$, as desired.
