Let $R$ be a ring.
We say that a function $f : R \to R$ is \emph{good} if for any $x, y \in R$,
\[ f(f(x) f(y)) + f(x + y) = f(xy). \tag{*}\label{2017a6-eq0} \]
We are interested in finding all good functions.
Unfortunately, this is too hard to solve in general.
We present a complete solution for three subcases.

\begin{theorem}\label{2017a6-1}
Let $R$ be a ring such that $2 \in R^{\times}$ and $3$ is not a zero divisor in $R$.
Let $(S, \phi, \iota, a)$ be a $4$-tuple with the following data:
\begin{itemize}
    \item   $S$ is a ring;
    \item   $\phi : R \to S$ is a ring homomorphism;
    \item   $\iota : S \to R$ is a group homomorphism with $\phi \circ \iota = \id_S$;
    \item   $a$ is an element of $Z(S)$ such that $a^2 = 1$.
\end{itemize}
Then $f(x) = \iota(a(1 - \phi(x)))$ works, and all good functions on $R$ arise from some $4$-tuple of the above form.
\end{theorem}

\begin{theorem}\label{2017a6-2}
Let $F$ be a division ring with $\rchar(F) \neq 2$.
Then the only good functions on $F$ are $0$, $x \mapsto 1 - x$, and $x \mapsto x - 1$.
\end{theorem}

\begin{theorem}\label{2017a6-3}
Let $F$ be a field with $\rchar(F) = 2$.
Then the only good functions on $F$ are $0$ and $x \mapsto x + 1$.
\end{theorem}

An example of "weird" good function that falls into the description in Theorem~\ref{2017a6-1} is
\[ R = R'[X], \quad f(P) = (1 - P(1)) X^k \;\; \forall P \in R'[X], \]
    where $R'$ is an arbitrary ring and $k \in \N$ is arbitrary.
Here, Theorem~\ref{2017a6-1} holds with $S = R'[X]/\langle X - 1\rangle \cong R'$, $\iota(r) = rX^k$, and $a = 1$.
The image of $f$ is the subgroup $\{rX^k : r \in R'\} \subseteq R'[X]$.
By changing it to another subgroup and also changing $S$ to another quotient of $R$, we could generate other good functions.
This should explain why the theorem is a reasonable guess in the first place in the general case.





\subsection*{Solution}

References:
\begin{itemize}
    
    \item
    \url{https://artofproblemsolving.com/community/c6h1480146p8693244}.
    
    Solution in AoPS by \textbf{anantmudgal09} (post \#75).

    \item
    \url{https://artofproblemsolving.com/community/c6h1480146p29214012}

    Solution in AoPS by \textbf{BlazingMuddy} (author of this project, post \#176).

\end{itemize}

The two references prove Theorem~\ref{2017a6-2} and Theorem~\ref{2017a6-3}, respectively.
Here, we build up the necessary theory of good functions and use it to prove all three main results.
Our proof of Theorem~\ref{2017a6-2} and Theorem~\ref{2017a6-3} uses this theory to simplify some arguments.

First, it is clear that $x \mapsto 1 - x$ is a good function.
For any good function $f : R \to R$ and $a \in Z(R)$ such that $a^2 = 1$, it can be checked that $x \mapsto a f(x)$ is also a good function.
To prove the easy direction of Theorem~\ref{2017a6-1}, it remains to show:

\begin{lemma}\label{2017a6-good-hom}
Let $\phi : R \to S$ is a ring homomorphism and $\iota : S \to R$ be a group homomorphism with $\phi \circ \iota = \id_S$.
For any $f : S \to S$, the function $\iota \circ f \circ \phi$ is good iff $f$ is good.
\end{lemma}
\begin{proof}
Since $\phi \circ \iota = \id_S$, we know that $\iota$ is injective and $\phi$ is surjective.
Thus, $\iota \circ f \circ \phi$ is good iff
\begin{align*}
    &\forall x, y \in R, \; \iota(f(f(\phi(x)) f(\phi(y)))) + \iota(f(\phi(x) + \phi(y))) = \iota(f(\phi(x) \phi(y))) \\
    \iff& \forall x, y \in R, \; f(f(\phi(x)) f(\phi(y))) + f(\phi(x) + \phi(y)) = f(\phi(x) \phi(y)) \\
    \iff& \forall x, y \in S, \; f(f(x) f(y)) + f(x + y) = f(xy),
\end{align*}
    iff $f$ is good.
\end{proof}

In all three main results, it remains to do the harder direction.
We start with the following big observation.

\begin{lemma}\label{2017a6-period-quot}
The period set $I = \{c \in R : \forall x \in R, f(x + c) = f(x)\}$ is a double-sided ideal.
The induced function $\tilde{f} : R/I \to R/I$ defined by $[x] \mapsto [f(x)]$ is good with no non-zero period.
\end{lemma}
\begin{proof}
Fix any $c \in I$.
Comparing~\eqref{2017a6-eq0} using $y = c$ and $y = 0$ gives $f(xc) = f(0)$ for all $x \in R$.
Then replacing $y$ with $yc$ yields
\[ f(x) + f(f(x) f(0)) = f(0) = f(xyc) = f(x + yc) + f(f(x) f(yc)) = f(x + yc) + f(f(x) f(0)), \]
    so $f(x + yc) = f(x)$ for all $x, y \in R$.
That is, $yc \in I$ for any $y \in R$.
Similarly, we also get $cy \in I$ for any $y \in R$.

Since $I$ is a double-sided ideal, there is an induced function $\tilde{f} : R/I \to R/I$ such that $\tilde{f}([x]) = [f(x)]$ for any $x \in R$.
Here, $[x]$ denotes the projection of $x \in R$ to $R/I$.
Now it is easy to check that $\tilde{f}$ is good.

Finally, we check that $\tilde{f}$ has no non-zero period.
For any $x, y \in R$, if $[f(x)] = [f(y)]$, then we get
\[ f(0) - f(x) = f(f(0) f(x)) = f(f(0) f(y)) = f(0) - f(y) \implies f(x) = f(y). \]
Now for any period $c \in R/I$, lift it arbitrarily to $\tilde{c} \in R$.
Then $\tilde{f}([x] + c) = \tilde{f}([x]) \iff f(x + \tilde{c}) = f(x)$ for any $x \in R$.
The latter yields $\tilde{c} \in I$ and thus $c = 0$.
\end{proof}

From now on, we say that $f : R \to R$ is \emph{reduced good} if $f$ is good and has no non-zero periods.
The above lemma means that in order to study good functions, we should study the reduced good ones.
We continue with easier observations.

First, plugging $x = y = 0$ into~\eqref{2017a6-eq0} yields $f(f(0)^2) = 0$.
Next, we prove:

\begin{lemma}\label{2017a6-map-eq-add-one-and-neg}
For any $a, b \in R$, if $f(a) = f(b)$, then $f(a + 1) = f(b + 1)$ and $f(-a) = f(-b)$.
\end{lemma}
\begin{proof}
Plugging $y = 1$ into~\eqref{2017a6-eq0} yields $f(x + 1) = f(x) - f(f(x) f(1))$ for any $x \in R$.
This clearly implies $f(a + 1) = f(b + 1)$.
Next, plugging $y = -1$ into~\eqref{2017a6-eq0} yields $f(-x) = f(x - 1) - f(f(x) f(-1))$ for any $x \in R$.
Since $f(a) = f(b)$ and $f(a + 1) = f(b + 1)$, we get $f(-(a + 1)) = f(-(b + 1))$.
Repeating the previous process once gives $f(-a) = f(-b)$.
\end{proof}

\begin{lemma}\label{2017a6-reduced-good-map-eq-zero}
If $f$ is reduced good, then $f(C) = 0 \iff C = 1$.
\end{lemma}
\begin{proof}
Since $f^{-1}(0)$ is non-empty, it suffices to show that for any $C \in R$, $f(C) = 0$ implies $C = 1$.
We start with $f(Cx) = f(0) + f(C + x)$ for all $x \in R$, obtained from~\eqref{2017a6-eq0}.
Then we get $f(C + 1) = -f(0)$ and $f(-C^2) = 2 f(0)$.
Plugging $x = 0$ and $y = -C^2$ into~\eqref{2017a6-eq0} yields $f(2 f(0)^2) = -f(0)$.
By Lemma~\ref{2017a6-map-eq-add-one-and-neg}, since $f(C) = f(f(0)^2) = 0$, we get $f(2C) = -f(0)$.

Now plugging $x = y = C$ into~\eqref{2017a6-eq0} yields $f(C^2) = 0$.
Plugging $x = C$ and $y = C^2$ yields $f(0) + f(C^2 + C) = f(C^3)$.
On the other hand, since $f(C) = 0$, we get $f(C^2 + C) = f(2C + 1) + f(0)$.
Since $f(C + 1) = f(2C) = -f(0)$, Lemma~\ref{2017a6-map-eq-add-one-and-neg} gives
\[ f(2C + 1) = f(C + 2) = f(2C) - f(0) = -2 f(0) \implies f(C^2 + C) = -f(0) \implies f(C^3) = 0. \]

Now we are ready for the final step.
For any $x \in R$, we write $f(C^4 x)$ in two ways, using the fact that $f(C) = f(C^2) = f(C^3) = 0$.
\[ f(C^4 x) = f(C^2 + C^2 x) + f(0) = f(C^2 (x + 1)) + f(0) = f(C^2 + x + 1) + 2 f(0), \]
\[ f(C^4 x) = f(C^3 + Cx) + f(0) = f(C (C^2 + x)) + f(0) = f(C^2 + x + C) + 2 f(0). \]
Replacing $x$ with $x - C^2$ gives $f(x + C) = f(x + 1)$ for any $x \in R$.
Since $f$ is reduced good, this forces $C = 1$, as desired.
\end{proof}

The proof alone implies that for any $f : R \to R$ good and $C \in f^{-1}(0)$, $C - 1$ is a period of $f$.
In particular, $f(1) = 0$ holds and $f(0)^2 - 1$ is a period of $f$.
Plugging $y = 1$ yields
\[ f(x + 1) + f(0) = f(x) \quad \forall x \in R. \tag{1}\label{2017a6-eq1} \]
For any ring $R$, let $Z(R) = \{c \in R : cx = xc \; \forall x \in R\}$ denote the centre of $R$.

\begin{lemma}\label{2017a6-good-injective}
Let $f : R \to R$ be an injective good function.
Then $f(0) \in Z(R)$, $f(0)^2 = 1$, and $f(x) = f(0) (1 - x)$ for any $x \in R$.
\end{lemma}
\begin{proof}
Since $f(1) = f(f(0)^2) = 0$, injectivity yields $f(0)^2 = 1$.
For any $x \in R$, we have $f(f(0) f(x)) = f(0) - f(x)$.
So $f(0) - f(x) \in f(R)$, and applying that to the same equation yields
\[ f(1 - f(0) f(x)) = f(f(0) (f(0) - f(x))) = f(x) \implies 1 - f(0) f(x) = x \iff f(x) = f(0) (1 - x). \]
Finally, note that $f(f(0) f(x)) = f(f(x) f(0)) = f(0) - f(x)$, so $f(0) f(x) = f(x) f(0)$ for any $x \in R$.
Thus $f(1 - x) = f(0) x = x f(0)$ for any $x \in R$, proving that $f(0) \in Z(R)$.
\end{proof}

\begin{lemma}\label{2017a6-reduced-good-NZD-two}
Let $R$ be a ring such that $2$ is not a zero divisor in $R$.
Let $f : R \to R$ be a reduced good function.
Then $f(0) \in Z(R)$, $f(0)^2 = 1$, and $f(x) = f(0) (1 - x)$ for any $x \in R$.
\end{lemma}
\begin{proof}
By Lemma~\ref{2017a6-good-injective}, it suffices to show that $f$ is injective.
Indeed, consider any $a, b \in R$ and suppose that $f(a) = f(b)$.
Then $f(a)$ and $f(b)$ commute, so $f(ab) = f(ba) = f(a + b) + f(f(a) f(b))$.
By Lemma~\ref{2017a6-map-eq-add-one-and-neg}, we get $f(-a) = f(-b)$ and $f(-ab) = f(-ba)$.
Hence, we get
\[ f(a - b) = f(-ab) - f(f(a) f(-b)) = f(-ba) - f(f(b) f(-a)) = f(b - a). \]

It now remains to show more generally that $f(-c) = f(c)$ implies $c = 0$.
Indeed, plugging $x = -1$ and $y = c$ into~\eqref{2017a6-eq0} yields
\[ f(f(-1) f(c)) + f(c - 1) = f(-c) \implies f(f(-1) f(c)) + f(c) + f(0) = f(c) \implies f(f(-1) f(c) - 1) = 0. \]
By Lemma~\ref{2017a6-reduced-good-map-eq-zero}, this yields $f(-1) f(c) = 2$.
However, $f(-1) = f(0) + f(0) = 2 f(0)$.
Since $2$ is not a zero divisor in $R$, we get $f(0) f(c) = 1$.
Since $f(0)^2 = 1$, this yields $f(c) = f(0) \iff f(c + 1) = 0 \iff c = 0$.
\end{proof}

At this point, Theorem~\ref{2017a6-2} has already been proved.
Indeed, if $F$ is a division ring, then the only double-sided ideals of $F$ are $(0)$ and $F$.
So, the good functions on $F$ are either zero or reduced good.
If $\rchar(F) \neq 2$, then $2$ is not a zero divisor in $F$ and $f(0)^2 = 1 \iff f(0) = \pm 1$.



\subsubsection*{Proof of Theorem~\ref{2017a6-1}}

Again, let $I$ be the set of periods of $f$.
We take $S = R/I$ and $\phi : R \to R/I$ to be the natural quotient map.
From now on, for any $x \in R$, we denote $[x] = \phi(x)$.

By Lemma~\ref{2017a6-period-quot}, the induced map $\tilde{f} : R/I \to R/I$ is reduced good.
Since $2 \in R^{\times}$, it is guaranteed that $2$ is not a zero divisor in $R/I$.
Thus, by Lemma~\ref{2017a6-reduced-good-NZD-two}, there exists $a \in Z(R/I)$ with $a^2 = 1$ such that $\tilde{f}(x) = a(1 - x)$ for any $x \in R/I$.
Equivalently, $[f(x)] = a(1 - [x])$ for any $x \in R$.
It remains to find the desired $\iota : S \to R$.

For any $x, y \in R$, we have $[f(x) f(y)] = [a (1 - x) a (1 - y)] = [(1 - x)(1 - y)]$.
Thus, $f(f(x) f(y)) = f((1 - x)(1 - y))$, and the original functional equation changes to
\[ f((1 - x) (1 - y)) + f(x + y) = f(xy). \]
We start by proving that the function $g(x) = f(1 - x)$ is a group homomorphism.
That is, $g$ is additive.

\begin{lemma}\label{2017a6-side-FE-solution}
Let $R$ be a ring and $(G, +)$ be an abelian group.
Suppose that $G$ is $2$- and $3$-torsion free.
Let $g : R \to G$ be a function such that
\[ g(x + y - xy) + g(1 - (x + y)) = g(1 - xy) \quad \forall x, y \in R. \tag{1.1}\label{2017a6-eq-thm1-1} \]
Then $g$ is a group homomorphism.
\end{lemma}
\begin{proof}
Plugging $x = y = 0$ yields $g(0) = 0$.
Next, plugging $y = 1$ and replacing $x$ with $-x$ yields
\[ g(x + 1) = g(x) + g(1) \quad \forall x \in R. \tag{1.2}\label{2017a6-eq-thm1-2} \]
Plugging $y = 0$ and then using~\eqref{2017a6-eq-thm1-2} yields
\[ g(-x) = -g(x) \quad \forall x \in R. \tag{1.3}\label{2017a6-eq-thm1-3} \]
After some rearrangement,~\eqref{2017a6-eq-thm1-1} now yields
\[ g(xy + x + y) = g(xy) + g(x + y) \quad \forall x, y \in R. \tag{1.4}\label{2017a6-eq-thm1-4} \]
Plugging $y = 1$ yields
\[ g(2x) = 2 g(x) \quad \forall x \in R. \tag{1.5}\label{2017a6-eq-thm1-5} \]
Then plugging $y = 2$ yields
\[ g(3x) = 3 g(x) \quad \forall x \in R. \tag{1.6}\label{2017a6-eq-thm1-6} \]
More generally, one can show that $g(nx) = n g(x)$ for any $n \in \N$ and $x \in R$, but the above two are enough.

... %% SOLVE THIS
\end{proof}

Now, the function $x \mapsto g(f(0) x)$ is a group homomorphism.
One can check that the kernel contains $I$ since $g(0) = 0$.
Thus there is an induced group homomorphism $\iota : S \to R$ with $\iota([x]) = g(f(0) x) = f(1 - f(0) x)$ for all $x \in R$.
We claim that this $\iota$ indeed works.

We first verify the form $f$ takes.
We have $[f(0)] = \tilde{f}(0) = a$.
Since $f(0)^2 - 1 \in I$, for any $x \in R$, we have
\[ \iota(a(1 - [x])) = f(1 - f(0) \cdot f(0) (1 - x)) = f(1 - (1 - x)) = f(x). \]
Finally, we check that $\phi \circ \iota = \id_S$.
Indeed, for any $x \in R$,
\[ [\iota([x])] = [f(1 - f(0) x)] = a(1 - [1 - f(0) x]) = a [f(0) x] = a^2 [x] = [x]. \]
This proves Theorem~\ref{2017a6-1}.



\subsubsection*{Proof of Theorem~\ref{2017a6-3}}

Let $F$ be a field of characteristic $2$.
As argued before, the good functions on $F$ are either zero or reduced good.
By Lemma~\ref{2017a6-good-injective}, it then suffices to show that reduced good functions over $F$ are injective.

Let $f : F \to F$ be a reduced good function.
Fix some $a, b \in F$ such that $f(a) = f(b)$.
By Lemma~\ref{2017a6-map-eq-add-one-and-neg}, we get $f(a + 1) = f(b + 1)$.
By Lemma~\ref{2017a6-reduced-good-map-eq-zero}, we have $a = 0 \iff b = 0$.
The remaining case is when $a$ and $b$ are non-zero.

We start with the following observation.
Consider~\eqref{2017a6-eq0} with $(x, y) = (a + 1, b^{-1} + 1)$.
Since $f(a + 1) = f(b + 1)$, applying~\eqref{2017a6-eq1} yields
\[ f((a + 1)(b^{-1} + 1)) = f((a + 1) + (b^{-1} + 1)) \implies f(ab^{-1} + a + b^{-1} + 1) = f(a + b^{-1}). \]
Then, by~\eqref{2017a6-map-eq-add-one-and-neg}, we get
\[ f(ab^{-1} + a + b^{-1}) = f(1 + a + b^{-1}). \]
By symmetry, we have
\[ f(ba^{-1} + b + a^{-1}) = f(1 + b + a^{-1}). \]
Now notice the identity
\[ (1 + a + b^{-1})(1 + b + a^{-1}) = (ab^{-1} + a + b^{-1})(ba^{-1} + b + a^{-1}). \]
As a result, by plugging the appropriate values into~\eqref{2017a6-eq0}, we get
\[ f((1 + a + b^{-1}) + (1 + b + a^{-1})) = f((ab^{-1} + a + b^{-1}) + (ba^{-1} + b + a^{-1})). \]
Letting $C = a + b + 1$ and $D = a^{-1} + b^{-1} + 1$, the above equation is equivalent to saying that
\[ f(C + D) = f(CD + 1) = f(CD) + 1. \]
But then plugging $(x, y) = (C, D)$ into~\eqref{2017a6-eq0} yields
\[ f(f(C) f(D)) = 1 \iff f(C) f(D) = 0 \iff f(C) = 0 \vee f(D) = 0 \iff C = 1 \vee D = 1. \]
By definition of $C$ and $D$, and by $\rchar(F) = 2$, this is equivalent to
\[ a + b + 1 = 1 \vee a^{-1} + b^{-1} + 1 = 1 \iff a = b \vee a^{-1} = b^{-1}. \]
But $a^{-1} = b^{-1}$ yields $a = b$.
So, regardless, we have obtained $a = b$.
This proves that $f$ is injective.
