A \emph{floor function} $\lfloor \cdot \rfloor : R \to \Z$ on a totally ordered ring $R$ is a function such that, for any $r \in R$ and $n \in \Z$, $n \leq \lfloor r \rfloor$ if and only if $n \leq r$ in $R$.

Let $R$ and $S$ be totally ordered rings with floor.
Find all functions $f : F \to R$ such that, for any $x, y \in R$,
\[ f(\lfloor x \rfloor y) = \lfloor f(y) \rfloor f(x). \tag{*}\label{2010a1-eq0} \]



\subsection*{Answer}

If $R \not\cong \Z$, then either $f \equiv 0$ or $f \equiv C$ for some $C \in S$ with $\lfloor C \rfloor = 1$.

If $R = \Z$, then either $f \equiv 0$ or one of the following:
\begin{itemize}
    
    \item
    $f$ is induced from a monoid homomorphism $g : \Z \to \Z$.

    \item
    There exists a divisor-closed, product-closed subset $M \subseteq \Z$ and $C \in S$ with $\lfloor C \rfloor = 1$ such that, for any $x \in R$,
    \[ f(x) = \begin{cases} C, & x \in M, \\ 0, & x \notin M. \end{cases} \]
    By divisor-closed, we mean that for any $n \in M$ and integer $k \mid n$, we have $k \in M$, even if $n = 0$.
    In particular, if $0 \in M$, then $M = \Z$.
    
    \item
    There exists a monoid homomorphism $\phi : \Z \to \N$ and an infinitesimal element $\epsilon \in R$ such that $f(x) = \phi(x)(1 + \epsilon)$.
    Here infinitesimal means that $0 \leq \epsilon$ and $k \epsilon < 1$ for all $k \in \N$.

\end{itemize}



\subsection*{Solution}

Plugging $(x, y) = (0, 0)$ into~\eqref{2010a1-eq0} yields $f(0) = \lfloor f(0) \rfloor f(0)$.
Thus either $\lfloor f(0) \rfloor = 1$ or $f(0) = 0$.
In the former case, plugging $y = 0$ into~\eqref{2010a1-eq0} yields that $f$ is constant.
Clearly the underlying constant $C$ is equal to $f(0)$ and we have obtained $\lfloor C \rfloor = 1$.

Plugging $(x, y) = (1, 1)$ into~\eqref{2010a1-eq0} yields $f(1) = \lfloor f(1) \rfloor f(1)$.
Thus either $\lfloor f(1) \rfloor = 1$ or $f(1) = 0$.
In the latter case, plugging $x = 1$ into~\eqref{2010a1-eq0} yields that $f \equiv 0$.
It now remains to solve for the case where $f(0) = 0$ and $\lfloor f(1) \rfloor = 1$.

We first show that there are no other possible functions $f$ when $R$ is not (isomorphic to) $\Z$.
In this case, we can show that there exists $c \in R$ such that $0 < c < 1$.
In particular, $\lfloor c \rfloor = 0$ and $\lfloor -c \rfloor = -1$.
Plugging $(x, y) = (c, 1)$ into~\eqref{2010a1-eq0} yields $f(c) = f(0) = 0$.
Plugging $x = -1$ into~\eqref{2010a1-eq0} yields $f(-y) = \lfloor f(y) \rfloor f(-1)$ for any $y \in R$.
In particular, since $f(c) = 0$, we get $f(-c) = 0$.
Plugging $(x, y) = (-c, 1)$ into~\eqref{2010a1-eq0} yields $f(-1) = 0$.
Thus $f(-y) = 0$ for all $y \in R$, and thus $f \equiv 0$.

It remains to consider the case where $R = \Z$.
The functional equation reduces to $f(xy) = \lfloor f(y) \rfloor f(x)$ for all $x, y \in \Z$.
Again, we can assume that $f(0) = 0$ and $\lfloor f(1) \rfloor = 1$.
For convenience, denote $g(x) = \lfloor f(x) \rfloor$ for any $x \in R$.
Plugging $x = 1$ into~\eqref{2010a1-eq0} yields
\[ f(x) = \lfloor f(x) \rfloor f(1) = g(x) f(1) \quad \forall x \in R. \tag{1}\label{2010a1-eq1} \]
Thus~\eqref{2010a1-eq0} becomes $g(xy) f(1) = g(x) g(y) f(1)$.
Since $\lfloor f(1) \rfloor = 1$, we have $f(1) \neq 0$, so $g(xy) = g(x) g(y)$ for all $x, y \in R$.
In other words, $g : \Z \to \Z$ is a homomorphism of multiplicative monoids.
We divide into several cases:

\begin{enumerate}

    \item 
    $g$ attains a negative value.

    Choose some $n \in \Z$ such that $\lfloor f(n) \rfloor < 0$.
    Since $f(1) \geq \lfloor f(1) \rfloor = 1$, we have
    \[ f(n) = \lfloor f(n) \rfloor f(1) \leq \lfloor f(n) \rfloor \leq f(n). \]
    The inequalities are all equalities, which forces $f(1) = 1$.
    Thus $f : \Z \to R$ is just a lift from a monoid homomorphism $g : \Z \to \Z$.

    \item
    $g$ only attains positive values and is bounded.

    Then one can easily show that $g$ only attains $0$ and $1$.
    Namely, if $g(n) > 1$ for some $n \in \Z$, the sequence $(g(n^k))_{k \geq 1} = (g(n)^k)_{k \geq 1}$ is unbounded.
    Then $g$ is an indicator function for the subset $M = g^{-1}(1) \subseteq \Z$.
    Note that $M$ is divisor-closed and product-closed, since $g$ is a monoid homomorphism.
    Finally, we can check that any value $C$ for $f(1)$ works as long as $\lfloor C \rfloor = 1$.

    \item
    $g$ only attains positive values and is unbounded.

    For any $k \in \N$, we can choose some $n \in \Z$ such that $\lfloor f(x) \rfloor \geq k$.
    Then we get
    \[ \lfloor f(x) \rfloor + 1 > f(x) = \lfloor f(x) \rfloor f(1) \iff \lfloor f(x) \rfloor (f(1) - 1) < 1. \]
    For convenience, denote $\epsilon = f(1) - 1$.
    Since $\epsilon \geq 0$, this implies $k \epsilon < 1$ for any $k \in \N$.
    That is, $\epsilon$ is either zero or an "infinitesimal" element.
    In summary, we get $f(x) = g(x) (1 + \epsilon)$ for all $x \in \Z$, with $g : \Z \to \N$ being a monoid homomorphism.

\end{enumerate}



\subsection*{Extra notes}

The right hand side in the original problem is $f(x) \lfloor f(y) \rfloor$, not $\lfloor f(y) \rfloor f(x)$.
However, despite $R$ being not necessarily commutative, both expressions are equal since integers commute with every element of $R$.

The result for the case $R = \Z$ is indeed the best we can do.
Each of the three non-trivial cases has at least one non-trivial example.
In the first case, for example, we can take $g(n) = c^{\nu_2(n)}$ for some integer $c$ for $n \neq 0$ (and $g(0) = 0$).
In the second case, the set $M$ that we can choose are precisely $\Z$ and those defined as the set of integers whose prime factors belong to a fixed subset of the primes.
In the third case, there exists a floor ring $S$ with a non-zero infinitesimal element $\epsilon$.
Indeed, take $S = \Z[X]$ with the ordering defined by $f(X) = a_0 + a_1 X + \ldots + a_n X^n > 0$ if and only if $a_i > 0$, where $i$ is the smallest index such that $a_i \neq 0$.
One can check that $S$ is a totally ordered ring and for any $f \in \Z[X]$, we can choose a unique $n \in \Z$ such that $n \leq f < n + 1$.
Namely, $n = f(0)$ if $f(X) - f(0) \geq 0$, and $n = f(0) - 1$ if $f(X) - f(0) < 0$.
